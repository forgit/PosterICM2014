\documentclass{tikzposter} %Options for format can be included here

 % Title, Author, Institute
\title{
\vbox{
Inverse problems for Sturm-Liouville operators  
with Bessel-type singularity inside an interval
}
}
\author{Alexey Fedoseev}
\institute{Saratov State University, Russia}
%\titlegraphic{LogoGraphic Inserted Here}

 %Choose Layout
\usetheme{Default}

\begin{document}

 % Title block with title, author, logo, etc.
\maketitle

 \block{What is Inverse Problems?}
{
Solution of an inverse problem entails determining unknown causes, based on observation of their effects.
}
 \begin{columns}

 % FIRST column
\column{0.6}% Width set relative to text width

\block{Large Column}{Text\\Text\\Text Text Text}
\note{Note with default behavior}
\note[targetoffsetx=12cm, targetoffsety=-1cm, angle=20, rotate=25]
{Note \\ offset and rotated}

 % First column - second block
\block{Block titles with enough text will automatically obey spacing requirements }
{Text\\Text}

 % First column - third block
\block{Sample Block 4}{T\\E\\S\\T}

 % SECOND column
\column{0.4}
 %Second column with first block's top edge aligned with with previous column's top.

 % Second column - first block
\block[titleleft]{Smaller Column}{Test}

 % Second column - second block
\block[titlewidthscale=0.6, bodywidthscale=0.8]
{Variable width title}{Block with smaller width.}

 % Second column - third block
\block{}{Block with no title}

 % Second column - A collection of blocks in subcolumn environment.
\begin{subcolumns}
    \subcolumn{0.27} \block{1}{First block.} \block{2}{Second block}
    \subcolumn{0.4} \block{Sub-columns}{Sample subblocks\\Second subcolumn}
    \subcolumn{0.33} \block{4}{Fourth} \block{}{Final Subcolumn block}
\end{subcolumns}

 % Bottomblock
\block{Final Block in column}{
    Sample block.
}
\end{columns}




%\block[titleleft, titleoffsetx=2em, titleoffsety=1em, bodyoffsetx=2em,%
% bodyoffsety=-2cm, roundedcorners=10, linewidth=0mm, titlewidthscale=0.7,%
% bodywidthscale=0.9, bodyverticalshift=2cm, titleright]
%{Block outside of Columns}{Along with several options enabled}

\end{document}



\endinput
%%
%% End of file `tikzposter-template.tex'.
